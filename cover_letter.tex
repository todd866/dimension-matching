\documentclass[11pt]{letter}
\usepackage[margin=1in]{geometry}
\usepackage{hyperref}

\signature{Ian Todd\\
Sydney Medical School\\
University of Sydney}

\address{Ian Todd\\
Sydney Medical School\\
University of Sydney\\
Sydney, NSW 2006, Australia\\
itod2305@uni.sydney.edu.au}

\begin{document}

\begin{letter}{Editorial Office\\
Chaos: An Interdisciplinary Journal of Nonlinear Science\\
AIP Publishing}

\opening{Dear Editors,}

I am pleased to submit the manuscript entitled ``\textbf{Dimension Matching in Multiscale Chaotic Systems: When Correlations and Spectra Coincide}'' for consideration as a Regular Article in \textit{Chaos}.

\textbf{Summary.} This paper establishes a framework for understanding when independently defined complexity measures---correlation dimension and Fourier dimension---agree in multiscale stochastic systems. Drawing on recent mathematical advances in Gaussian multiplicative chaos (GMC), including the recently-proven Lin-Qiu-Tan result (arXiv:2411.13923) confirming the Garban-Vargas conjecture, I show that dimension matching reflects a coherence condition: cross-scale consistency maintained by martingale-like balance. When this consistency holds, low-dimensional projections faithfully represent high-dimensional dynamics; when it fails near the critical threshold $\gamma = \sqrt{2}$, such projections become unreliable.

\textbf{Key contributions:}
\begin{enumerate}
    \item I interpret the recently-proven Garban-Vargas conjecture (on GMC dimension matching) as a coherence condition with potential applicability to chaotic systems beyond the GMC setting.

    \item I develop a game-theoretic interpretation where dimension matching corresponds to a cooperative equilibrium among scales---no scale can ``defect'' profitably from the martingale budget.

    \item I propose that dimension matching breakdown may serve as a diagnostic for regime transitions, with potential applications to neural dynamics and multi-agent systems.

    \item I clarify the distinction between coherence (high-dimensional constraint maintenance) and phase-locking (dimensional collapse), showing these are fundamentally different phenomena.
\end{enumerate}

\textbf{Relevance to Chaos.} This work directly addresses core concerns of nonlinear dynamics: when do simplified descriptions capture essential structure, and when do they fail? The dimension-matching framework provides both theoretical insight (connecting geometry and harmonics via martingales) and practical diagnostics (measurable early-warning indicators for transitions). The game-theoretic framing offers a novel perspective on multiscale coordination that should interest researchers across the journal's interdisciplinary scope.

\textbf{Technical validation.} The paper includes numerical simulations of GMC measures, agent-based coordination games, and synthetic neural time series analysis. All code is available for reproducibility.

This manuscript has not been published elsewhere and is not under consideration by another journal.

I suggest the following potential reviewers with relevant expertise:
\begin{itemize}
    \item Vincent Vargas (University of Geneva) --- co-author of the Garban-Vargas conjecture
    \item Dietmar Plenz (NIH) --- neural criticality and multiscale dynamics
    \item James Crutchfield (UC Davis) --- computational mechanics and complexity measures
\end{itemize}

Thank you for considering this submission.

\closing{Sincerely,}

\end{letter}
\end{document}
