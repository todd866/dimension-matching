\documentclass[11pt]{letter}
\usepackage[margin=1in]{geometry}
\usepackage{hyperref}

\signature{Ian Todd\\Sydney Medical School\\University of Sydney}
\address{Sydney Medical School\\University of Sydney\\Sydney, NSW 2006, Australia\\itod2305@uni.sydney.edu.au}

\begin{document}

\begin{letter}{Editorial Office\\Physics of Life Reviews}

\opening{Dear Editors,}

I am pleased to submit ``The Geometry of Rhythms: Dimension Matching as a Spectral-Geometric Constraint on Living Matter'' for consideration as a review/perspective article.

\textbf{The central question:} What mathematically distinguishes living matter from dead matter---and from engineered computation? Living systems maintain coordinated dynamics across multiple scales; death represents the collapse of this coordination. But we lack a precise framework for this intuition that goes beyond thermodynamics.

\textbf{The proposal:} I argue that life is characterized by \emph{cross-scale coherence}: agreement among independent complexity estimators measuring geometric (state-space) and spectral (oscillatory) structure. Drawing on Takens' embedding theorem, Gaussian multiplicative chaos, and constraint closure theory, I propose that such estimator agreement is the measurable signature of viable biological organization---and that its breakdown signals pathology or death.

\textbf{Key contributions:}
\begin{itemize}
\item A mathematical toolkit for measuring cross-scale coherence, synthesizing dynamical systems theory, probability, and information geometry
\item A game-theoretic interpretation grounding the framework in evolutionary dynamics
\item Application to the biological vs. artificial intelligence distinction: current AI lacks intrinsic temporal dynamics, which may explain its brittleness compared to biological cognition
\item Concrete experimental pipelines using existing clinical data (HRV, EEG)
\item Evidence from dormant life (tardigrades, spores) that geometric coherence---not thermodynamic activity---defines the life/death boundary
\end{itemize}

\textbf{What the paper is and isn't:} This is a hypothesis paper, not a proof. I am explicit about measurement limitations and what would constitute evidence. The goal is to articulate a precise, falsifiable framework that unifies observations across physiology, neuroscience, and theoretical biology.

The manuscript has not been submitted elsewhere and presents original synthesis. I believe it fits the interdisciplinary mandate of \textit{Physics of Life Reviews} and would stimulate discussion across fields.

\closing{Sincerely,}

\end{letter}
\end{document}
