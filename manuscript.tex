\documentclass[12pt]{article}
\usepackage[utf8]{inputenc}
\usepackage[T1]{fontenc}
\usepackage{lmodern}
\usepackage{amsmath}
\usepackage{amssymb}
\usepackage{amsthm}
\usepackage{geometry}
\geometry{margin=1in}
\usepackage{setspace}
\usepackage{hyperref}
\usepackage{graphicx}
\usepackage{float}
\usepackage{booktabs}
\usepackage[round]{natbib}
\bibliographystyle{plainnat}
\doublespacing

\newtheorem{theorem}{Theorem}
\newtheorem{lemma}[theorem]{Lemma}
\newtheorem{definition}[theorem]{Definition}
\newtheorem{corollary}[theorem]{Corollary}
\newtheorem{proposition}[theorem]{Proposition}

\title{Dimension Matching in Multiscale Chaotic Systems: \\
When Correlations and Spectra Coincide}

\author{Ian Todd\\
Sydney Medical School, University of Sydney\\
Sydney, NSW 2006, Australia\\
\texttt{itod2305@uni.sydney.edu.au}}

\date{December 2025}

\begin{document}

\maketitle

% LEAD PARAGRAPH (Chaos requirement - will appear in boldface)
\noindent\textbf{%
When can we trust a low-dimensional summary of a high-dimensional chaotic system? We show that ``dimension matching''---the agreement between geometric (correlation) and spectral (Fourier) complexity measures---provides a sharp criterion. In multiscale stochastic systems, these independently defined dimensions coincide exactly when a martingale-like balance holds across scales. This balance can be interpreted as a cooperative equilibrium: no scale ``cheats'' the cascade. At a critical threshold, the balance breaks, the dimensions decouple, and low-dimensional projections become unreliable. This framework offers a diagnostic for regime collapse applicable to neural dynamics, turbulence, and multi-agent coordination.
}

\vspace{1em}

\begin{abstract}
In multiscale stochastic systems, two fundamentally different notions of complexity—correlation dimension (measuring geometric clumpiness) and Fourier dimension (measuring spectral decay)—can be independently defined. Recent mathematical work on Gaussian multiplicative chaos (GMC) establishes that these dimensions provably coincide below a critical threshold ($\gamma < \sqrt{2}$), but decouple at the phase transition where the system collapses. We argue this ``dimension matching'' phenomenon reflects a coherence condition: cross-scale consistency maintained by martingale-like conservation laws. When the consistency holds, multiple distinct projections of the high-dimensional dynamics agree; when it fails, low-dimensional summaries become unreliable. We develop a game-theoretic interpretation where dimension matching corresponds to an incentive-compatible multiscale contract—no scale can profitably ``defect'' from the cascade. The breakdown of matching provides a diagnostic for regime transitions, with potential applications to neural dynamics, cellular coordination, and multi-agent systems. Our framework clarifies when coarse-grained observations faithfully represent high-dimensional chaotic structure, and when they must fail.

\textbf{Keywords:} dimension matching; Gaussian multiplicative chaos; multiscale dynamics; phase transitions; coherence; correlation dimension; Fourier dimension
\end{abstract}

%==============================================================================
\section{Introduction}
%==============================================================================

The world appears orderly, yet randomness and chaos pervade natural systems at every scale. Understanding when and how structure persists despite fluctuations—and when it collapses—remains a central challenge in nonlinear dynamics. This paper addresses a specific manifestation of this problem: when do independently defined measures of complexity agree, and what does their agreement (or disagreement) reveal about the underlying dynamics?

We focus on \emph{multiscale} chaotic systems: dynamics exhibiting structure at many nested scales simultaneously, as seen in turbulence cascades, neural avalanches, and multiplicative stochastic processes. For such systems, there exist at least two natural ways to quantify complexity:

\begin{enumerate}
    \item \textbf{Correlation dimension} ($D_C$): A geometric measure capturing how probability mass clusters across scales—the ``clumpiness'' of the system's distribution in state space.

    \item \textbf{Fourier dimension} ($D_F$): A spectral measure capturing how Fourier coefficients decay with frequency—the ``oscillatory richness'' of the dynamics.
\end{enumerate}

These arise from different mathematical traditions (geometry and probability vs.\ harmonic analysis) and need not agree \emph{a priori}. Yet recent work on Gaussian multiplicative chaos (GMC) establishes a remarkable result: in the subcritical regime, $D_C = D_F$ exactly \citep{garban2023, lin2024}. The dimensions match—until a phase transition, at which point the equivalence breaks and the system collapses.

This paper develops three claims:

\begin{enumerate}
    \item \textbf{Dimension matching as coherence}: The coincidence of $D_C$ and $D_F$ reflects a \emph{cross-scale consistency} condition—what we call coherence. This is not phase-locking (which reduces dimensionality) but the maintenance of structured relationships across many degrees of freedom.

    \item \textbf{Collapse as diagnostic}: The breakdown of dimension matching signals proximity to criticality. This provides a measurable early-warning indicator for regime shifts in complex systems.

    \item \textbf{Game-theoretic interpretation}: The subcritical regime can be understood as a multi-level coordination game where each scale is a ``player.'' Dimension matching holds when no scale can profitably defect from the cooperative equilibrium.
\end{enumerate}

We do not claim that biological or physical systems literally instantiate GMC. Rather, we use GMC as a mathematically tractable template that makes explicit what ``coherent multiscale structure'' means and when it must fail.

%==============================================================================
\section{Background: Dimensions, Information, and Chaos}
%==============================================================================

\subsection{Fractal Dimensions and Complexity}

The connection between dimension and complexity has a long history in dynamical systems theory. The \emph{correlation dimension} $D_2$, introduced by Grassberger and Procaccia, measures how the correlation integral $C(\varepsilon)$ scales with distance threshold $\varepsilon$:
\begin{equation}
    D_2 = \lim_{\varepsilon \to 0} \frac{\log C(\varepsilon)}{\log \varepsilon}
\end{equation}
where $C(\varepsilon)$ counts pairs of points within distance $\varepsilon$ \citep{grassberger1983}. This geometric quantity captures ``how much space the attractor fills'' at each scale.

More generally, the R\'{e}nyi dimensions $D_q$ form a spectrum parameterized by order $q$:
\begin{equation}
    D_q = \frac{1}{q-1} \lim_{\varepsilon \to 0} \frac{\log \sum_i p_i^q}{\log \varepsilon}
\end{equation}
where $p_i$ is the probability mass in the $i$-th box of a $\varepsilon$-partition. The information dimension $D_1$ (the $q \to 1$ limit) connects directly to Shannon entropy scaling.

\subsection{Dimension Is Not Information}

A crucial distinction: \emph{dimension characterizes geometric scaling, while information characterizes description length}. R\'{e}nyi's information dimension links them by defining dimension from how entropy grows under finer quantization:
\begin{equation}
    D_1 = \lim_{\varepsilon \to 0} \frac{H(X_\varepsilon)}{\log(1/\varepsilon)}
\end{equation}
But this does not identify the two concepts. Systems can be:
\begin{itemize}
    \item \textbf{High-dimensional yet low-entropy}: A system exploring a high-dimensional space may be confined to a thin manifold or few metastable basins, yielding low entropy rate despite large ambient dimension.

    \item \textbf{Low-dimensional yet high-entropy}: A 1D or 2D chaotic map can have high entropy rate (rapid information production) despite low attractor dimension.
\end{itemize}

Modern complexity frameworks make this separation explicit by contrasting geometric measures (fractal dimensions $D_q$) with computational measures (entropy rate, excess entropy, statistical complexity) \citep{crutchfield1989, feldman2008}.

\begin{figure}[H]
\centering
\includegraphics[width=0.75\textwidth]{figures/fig5_info_vs_dimension.png}
\caption{Information and dimension are distinct axes of complexity. A system can be high-dimensional yet low-entropy (confined to a thin manifold), low-dimensional yet high-entropy (rapid information production in a low-D chaotic map), or exhibit ``coherent chaos'' where both are high and dimension matching holds. The trajectory from simple dynamics to coherent complexity follows the dimension-matching regime; collapse pushes toward pure noise where structure is lost.}
\label{fig:info_dim}
\end{figure}

\subsection{Gaussian Multiplicative Chaos}

Gaussian multiplicative chaos (GMC) provides a rigorous framework for studying randomness that persists across all scales. First developed by Kahane in 1985 and revived in modern probability theory, GMC constructs random measures from log-correlated Gaussian fields \citep{kahane1985, rhodes2014}.

The construction proceeds as follows. Let $X$ be a log-correlated Gaussian field on a domain $D$ (e.g., the circle or plane), meaning its covariance behaves like:
\begin{equation}
    \mathbb{E}[X(x) X(y)] \sim -\log|x - y| \quad \text{as } |x-y| \to 0
\end{equation}
The GMC measure is formally $\mu_\gamma = e^{\gamma X - \frac{\gamma^2}{2}\mathbb{E}[X^2]} dx$, with regularization needed to make sense of the exponential of a distribution.

The parameter $\gamma \in [0, \sqrt{2})$ controls the ``strength'' of the chaos:
\begin{itemize}
    \item \textbf{Subcritical} ($\gamma < \sqrt{2}$): The measure is well-defined, multifractal, and captures structured fluctuations at every scale.

    \item \textbf{Critical} ($\gamma = \sqrt{2}$): The measure collapses—mass concentrates on a set of zero Lebesgue measure.

    \item \textbf{Supercritical} ($\gamma > \sqrt{2}$): The naive construction fails entirely; modified definitions are required.
\end{itemize}

GMC appears in turbulence modeling, quantum gravity (Liouville theory), random matrix theory, and the statistical mechanics of disordered systems. Its universality makes it a natural template for multiscale chaos.

\begin{figure}[H]
\centering
\includegraphics[width=\textwidth]{figures/fig1_gmc_measures.png}
\caption{Gaussian multiplicative chaos measures at increasing values of the parameter $\gamma$. As $\gamma$ increases toward the critical value of $\sqrt{2} \approx 1.41$, the measure becomes increasingly concentrated (spiky), with mass condensing onto smaller regions. The participation ratio (PR) quantifies this concentration: higher values indicate more uniform distribution, while lower values indicate mass concentration. In the subcritical regime ($\gamma < \sqrt{2}$), the measure remains well-defined; at criticality, it collapses.}
\label{fig:gmc}
\end{figure}

%==============================================================================
\section{The Dimension Matching Phenomenon}
%==============================================================================

\subsection{Two Notions of Dimension for GMC}

For GMC measures $\mu_\gamma$ on the circle, Garban and Vargas identified two independently defined dimensions \citep{garban2023}:

\begin{definition}[Correlation Dimension / $L^2$-spectrum]
The correlation dimension $D_C(\gamma)$ (equivalently, the $L^2$-spectrum $\dim_2$) characterizes the integrability of pairwise correlations:
\begin{equation}
    \iint |x-y|^{-s} \, d\mu_\gamma(x) \, d\mu_\gamma(y) < \infty
\end{equation}
for $s < D_C(\gamma)$, where $D_C$ is the largest exponent for which this integral converges.
\end{definition}

\begin{definition}[Fourier Dimension]
The Fourier dimension $D_F(\gamma)$ characterizes the decay rate of Fourier coefficients:
\begin{equation}
    |\hat{\mu}_\gamma(n)| = O(|n|^{-s/2}) \quad \text{as } |n| \to \infty
\end{equation}
where $D_F$ is the supremum of all $s$ for which this decay holds.
\end{definition}

These dimensions emerge from completely different calculations—one geometric (how probability mass clusters), one spectral (how fast Fourier coefficients decay). There is no obvious reason they should agree.

\subsection{The Garban-Vargas Conjecture and Its Resolution}

In 2023, Garban and Vargas conjectured that the Fourier dimension of GMC equals the correlation dimension, and gave the explicit formula \citep{garban2023}:

\begin{theorem}[Dimension Matching, \citealt{lin2024}]
\label{thm:matching}
For Gaussian multiplicative chaos with parameter $\gamma \in (0, \sqrt{2})$:
\begin{equation}
    D_C(\gamma) = D_F(\gamma) = D^*(\gamma)
\end{equation}
where the common dimension $D^*(\gamma)$ is given by:
\begin{equation}
    D^*(\gamma) = \begin{cases}
        1 - \gamma^2 & \text{if } 0 < \gamma < 1/\sqrt{2} \\[4pt]
        (\sqrt{2} - \gamma)^2 & \text{if } 1/\sqrt{2} \leq \gamma < \sqrt{2}
    \end{cases}
\end{equation}
The two independently defined dimensions coincide exactly throughout the subcritical regime.
\end{theorem}

The piecewise nature of the formula reflects a transition at $\gamma = 1/\sqrt{2} \approx 0.707$: below this threshold, fluctuations are weak enough that the dimension decreases quadratically with $\gamma^2$; above it, the approach to criticality at $\gamma = \sqrt{2}$ dominates.

Lin, Qiu, and Tan proved this conjecture by revealing the mechanism: GMC possesses a \emph{vector-valued martingale structure} across scales \citep{lin2024}. A martingale is a ``fair game''—the expected value at finer scales equals the value at coarser scales, with no systematic drift. For GMC, this means fluctuations are ``conserved'' across the cascade: each scale contributes variance in a balanced way.

The martingale structure forces a conservation law that couples geometric and spectral properties. Because the same conservation law governs both, the dimensions must agree.

\subsection{Mathematical Formalization}

The proof of dimension matching relies on a sophisticated martingale analysis. For a measure $\mu$ on the circle $\mathbb{T}$:

\begin{definition}[$L^2$-spectrum and Fourier Dimension]
The \emph{$L^2$-spectrum} (or correlation dimension) is:
\begin{equation}
    \dim_2(\mu) = \sup\left\{ s : \iint |x - y|^{-s} \, d\mu(x) \, d\mu(y) < \infty \right\}
\end{equation}
The \emph{Fourier dimension} is:
\begin{equation}
    \dim_F(\mu) = \sup\left\{ s : |\hat{\mu}(n)| = O(|n|^{-s/2}) \right\}
\end{equation}
where $\hat{\mu}(n) = \int_{\mathbb{T}} e^{-2\pi i n x} \, d\mu(x)$.
\end{definition}

In general $\dim_F \leq \dim_2$, with equality characterizing ``Salem measures.'' The Garban-Vargas conjecture asserts that GMC is almost surely Salem:

\begin{proposition}[Dimension Formulas for GMC, \citealt{lin2024}]
For the GMC measure $\mu_\gamma$ with $\gamma \in (0, \sqrt{2})$:
\begin{equation}
    \dim_2(\mu_\gamma) = \dim_F(\mu_\gamma) = D^*(\gamma) = \begin{cases}
        1 - \gamma^2 & \gamma < 1/\sqrt{2} \\
        (\sqrt{2} - \gamma)^2 & \gamma \geq 1/\sqrt{2}
    \end{cases}
\end{equation}
Both dimensions decrease from $1$ at $\gamma = 0$ to $0$ at $\gamma = \sqrt{2}$.
\end{proposition}

\begin{lemma}[Vector-Valued Martingale Structure]
The GMC construction via Bacry-Muzy decomposition yields a sequence of $\ell^q$-valued martingales. The uniform $L^p(\ell^q)$-boundedness of these martingales, established via Pisier's martingale inequalities, forces optimal Fourier decay rates.
\end{lemma}

The proof proceeds by showing that the same martingale bounds control both the correlation integral and Fourier coefficient decay, forcing their equality throughout the subcritical regime.

The critical exponent $\gamma = \sqrt{2}$ marks the boundary where the martingale ceases to be uniformly integrable. Beyond this point, mass concentrates on atoms and dimension matching breaks down.

\subsection{Collapse at Criticality}

At $\gamma = \sqrt{2}$, the martingale balance breaks. Variance explodes, extremes dominate, and the scale-by-scale conservation fails. Concretely:
\begin{itemize}
    \item The GMC measure collapses to zero (the limiting measure vanishes)
    \item Correlations become singular
    \item Spectral properties diverge
    \item The dimension matching relationship $D_C = D_F$ breaks down
\end{itemize}

This phase transition marks the boundary between ``coherent chaos'' (structured randomness with well-defined projections) and ``pathological collapse'' (where coarse-grained descriptions fail).

\begin{figure}[H]
\centering
\includegraphics[width=\textwidth]{figures/fig2_dimension_matching.png}
\caption{Dimension matching in GMC simulations. (A) Correlation dimension $D_C$ and Fourier dimension $D_F$ estimates decrease as $\gamma$ increases, tracking each other throughout the subcritical regime. Near the critical point ($\gamma \to \sqrt{2}$), both approach zero as the measure collapses. The solid curve shows the theoretical $D^*(\gamma)$. (B) Scatter plot of $D_C$ vs.\ $D_F$ across $\gamma$ values, showing approximate agreement (points near the diagonal). \textbf{Note:} These are finite-resolution numerical estimates that illustrate qualitative trends; they are not intended as rigorous verification of the asymptotic theorem, which is proven analytically in \citet{lin2024}.}
\label{fig:matching}
\end{figure}

%==============================================================================
\section{Coherence as Cross-Scale Consistency}
%==============================================================================

\subsection{What Dimension Matching Reveals}

We interpret Theorem~\ref{thm:matching} as a \emph{coherence condition}. The coincidence of $D_C$ and $D_F$ means that different ways of observing the system—geometric vs.\ spectral—yield consistent answers about its complexity. This is nontrivial: for generic stochastic processes, these dimensions can differ arbitrarily.

The agreement reflects an underlying structural property: the system maintains consistent relationships across scales. We call this \emph{cross-scale coherence}.

\begin{definition}[Cross-Scale Coherence]
A multiscale stochastic system exhibits cross-scale coherence if independently defined complexity measures (geometric, spectral, entropic) yield equivalent scalings. Operationally: different projections of the high-dimensional dynamics agree.
\end{definition}

\subsection{Coherence Is Not Phase-Locking}

A crucial distinction: cross-scale coherence is \emph{not} the same as phase-locking or synchronization.

\textbf{Phase-locking} typically \emph{reduces} effective dimensionality: many components become entrained to a common rhythm, collapsing the system onto a low-dimensional manifold. This is sometimes called dimensional collapse.

\textbf{Cross-scale coherence} maintains high-dimensional structure: many constraints persist simultaneously, and the system explores its full state space—but does so in a consistent, structured way.

The triad is:
\begin{enumerate}
    \item \textbf{Incoherence}: No reliable constraints; dimensions disagree; projections unreliable
    \item \textbf{Coherence}: High-dimensional constraints maintained; dimensions agree; projections consistent
    \item \textbf{Collapse}: One mode/scale dominates; dimensions may trivially agree (both zero) but structure is lost
\end{enumerate}

Phase-locking lives between (2) and (3): it can stabilize the system but at the cost of dimensional richness. The ``sweet spot'' for complex behavior is coherence without collapse.

\subsection{Martingale Balance as ``Fairness''}

The martingale structure underlying dimension matching has an intuitive interpretation: no scale ``cheats'' the cascade.

In a martingale, knowledge at coarse scales gives the best prediction of finer scales—there's no systematic advantage to be gained by refining observation. For GMC, this means:
\begin{itemize}
    \item Small eddies don't systematically steal energy from large ones
    \item Fine-scale fluctuations don't overwhelm coarse-scale structure
    \item The cascade is ``fair'' in a precise probabilistic sense
\end{itemize}

When this fairness breaks (at criticality), one scale dominates and the entire multiscale structure degenerates.

\begin{figure}[H]
\centering
\includegraphics[width=0.85\textwidth]{figures/fig3_phase_diagram.png}
\caption{Conceptual phase diagram for multiscale chaotic systems. Three regimes are distinguished: (1) \emph{Incoherent}, where coupling is too weak for reliable cross-scale structure; (2) \emph{Coherent}, where dimension matching holds ($D_C = D_F$) and projections faithfully represent high-dimensional dynamics; (3) \emph{Collapsed}, beyond the critical transition, where one scale dominates and multiscale structure degenerates. The phase transition (red line) marks the boundary where martingale balance breaks and coherence is lost.}
\label{fig:phase}
\end{figure}

%==============================================================================
\section{Game-Theoretic Interpretation}
%==============================================================================

The martingale/fairness interpretation suggests a natural game-theoretic framing: multiscale dynamics as a coordination game among scales.

\subsection{Scales as Players}

Consider a system with structure at $L$ nested scales (or a continuum). At each scale $\ell$, there is an ``agent'' (not literal, but a useful abstraction) controlling:
\begin{itemize}
    \item How much variance/energy/influence to express at that scale
    \item How to interact with adjacent scales
\end{itemize}

The ``payoff'' for each scale-agent depends on:
\begin{itemize}
    \item Local amplification: expressing more variance locally
    \item Global stability: the system not collapsing
\end{itemize}

This is a classic stability-exploitation tradeoff.

\subsection{Dimension Matching as Equilibrium}

\begin{proposition}[Informal]
Dimension matching ($D_C = D_F$) corresponds to a cooperative equilibrium where no scale can unilaterally improve its payoff by deviating from the martingale budget.
\end{proposition}

In the subcritical regime:
\begin{itemize}
    \item Each scale contributes variance according to a fixed ``budget''
    \item Deviation (grabbing more variance) is unprofitable because it triggers instability
    \item The Nash equilibrium is the martingale-balanced cascade
\end{itemize}

At criticality:
\begin{itemize}
    \item The incentive structure changes
    \item Defection becomes profitable (or unavoidable)
    \item One or more scales ``win'' by concentrating mass
    \item The cooperative structure collapses
\end{itemize}

\subsection{Implications}

This framing suggests:
\begin{enumerate}
    \item \textbf{Design principle}: To maintain coherent multiscale dynamics, ensure no scale has an incentive to dominate. This is a constraint on coupling architectures.

    \item \textbf{Diagnostic}: Dimension decoupling signals that the ``game'' is leaving the cooperative regime—an early warning of collapse.

    \item \textbf{Control target}: Interventions should restore martingale balance, not simply suppress one scale.
\end{enumerate}

\begin{figure}[H]
\centering
\includegraphics[width=\textwidth]{figures/fig4_game_schematic.png}
\caption{Game-theoretic interpretation of dimension matching. (A) In the \emph{cooperative regime}, scales interact symmetrically, each contributing variance according to a martingale-balanced budget. No scale dominates; the system maintains coherent multiscale structure and dimension matching holds. (B) In the \emph{collapsed regime}, one scale ``defects'' and concentrates mass, breaking the cooperative equilibrium. The symmetric structure degenerates, coherence is lost, and dimension matching fails.}
\label{fig:game}
\end{figure}

\begin{figure}[H]
\centering
\includegraphics[width=\textwidth]{figures/fig6_game_dynamics.png}
\caption{Simulation of multiscale coordination game dynamics. Each panel shows learning dynamics for a different stability penalty weight. \textbf{Left} (low stability weight): The system collapses---one scale dominates and dimension matching fails. \textbf{Center} (moderate weight): Transition regime with fluctuating allocations. \textbf{Right} (high stability weight): The system converges to uniform (martingale-balanced) allocation, maintaining dimension matching. Bottom row shows the dimension matching error $|D_C - D_F|$ over time; coherent regimes maintain low error.}
\label{fig:game_dynamics}
\end{figure}

%==============================================================================
\section{Applications}
%==============================================================================

\textbf{A note on proxies.} The rigorous dimension matching result ($D_C = D_F$) applies specifically to GMC measures. For empirical applications, we cannot compute true correlation or Fourier dimensions from finite time series. Instead, we use \emph{proxy measures}: participation ratio for geometric complexity, spectral entropy for harmonic complexity. The hypothesis is that \emph{if} a biological system exhibits GMC-like multiscale structure, then proxy agreement should correlate with proximity to coherent regimes. The following applications are exploratory and should be understood as demonstrations of the conceptual framework, not rigorous tests of the GMC theorem.

\subsection{Neural Dynamics}

Neural systems exhibit multiscale structure: from ion channels to synapses to microcircuits to brain regions. The ``criticality hypothesis'' suggests that healthy brain dynamics operate near a phase transition \citep{beggs2003, shew2013}.

Dimension matching suggests a testable prediction:
\begin{itemize}
    \item In healthy/awake states, geometric and spectral complexity proxies should show better agreement
    \item Near pathological transitions (seizure, anesthesia), the proxies may decouple
    \item The pattern of decoupling may indicate which scales dominate
\end{itemize}

Figure~\ref{fig:neural} demonstrates this prediction using synthetic EEG data calibrated to different brain states. The key finding: proxy agreement (low error between participation ratio and spectral entropy) is highest in the awake state and degrades significantly during seizure, consistent with the hypothesis that pathological hypersynchrony disrupts the cross-scale balance.

\begin{figure}[H]
\centering
\includegraphics[width=\textwidth]{figures/fig7_neural_analysis.png}
\caption{Complexity proxy agreement in simulated neural dynamics across brain states. (A) Example EEG traces showing characteristic patterns: awake (rich, variable), sleep (slower, more coherent), seizure (hypersynchronous), anesthesia (suppressed). (B) Geometric complexity proxy (participation ratio) vs.\ spectral complexity proxy, showing that awake dynamics cluster in the high-complexity region while pathological states show reduced dimensionality. (C) Match error (relative difference between complexity proxies) is lowest for awake states and highest for seizure, consistent with the prediction that coherent dynamics maintain better proxy agreement. (D) Summary showing that seizure and anesthesia exhibit significantly worse proxy agreement than awake states ($p < 0.001$, Mann-Whitney U test). Note: these are proxy measures, not true GMC dimensions.}
\label{fig:neural}
\end{figure}

\subsection{Cellular Systems}

Cells exhibit cascading fluctuations in gene expression, calcium signaling, and collective migration. The same framework applies:
\begin{itemize}
    \item Healthy tissue: coordinated fluctuations across scales (dimension matching)
    \item Stress/pathology: one scale dominates (dimension decoupling)
    \item Cancer: loss of multiscale coherence, reversion to autonomous single-scale dynamics
\end{itemize}

\subsection{Multi-Agent Systems}

For engineered multi-agent systems (swarms, distributed computing, economic networks):
\begin{itemize}
    \item Dimension matching indicates robust coordination
    \item Decoupling predicts cascade failures
    \item The martingale condition becomes a design constraint on agent interactions
\end{itemize}

%==============================================================================
\section{Discussion}
%==============================================================================

\subsection{What We Claim (and Don't)}

We do not claim that GMC is the unique or correct model for biological or physical systems. Rather, we argue:
\begin{enumerate}
    \item Dimension matching is a \emph{generic} phenomenon for multiscale systems with martingale-like balance
    \item Its breakdown signals approaching criticality
    \item The framework provides \emph{operational} metrics: estimate geometric and spectral complexity from data and compare
\end{enumerate}

The specific formula $D_C = D_F$ holds exactly for GMC. For other systems, we expect approximate matching in coherent regimes and systematic deviation near transitions.

\subsection{Relation to Existing Frameworks}

Our approach connects several existing ideas:
\begin{itemize}
    \item \textbf{Multifractal analysis}: The R\'{e}nyi dimension spectrum $D_q$ is a standard tool; we add the spectral dimension and their comparison.

    \item \textbf{Criticality hypothesis}: We sharpen ``near criticality'' to mean ``near dimension-matching breakdown.''

    \item \textbf{Effective dimensionality}: Participation ratios and related measures assess how many modes are active; we add the constraint that different methods should agree.

    \item \textbf{Computational mechanics}: We complement the geometry/spectral view with entropy-based complexity, but the key insight is about \emph{agreement} between measures.
\end{itemize}

\subsection{Open Questions}

\begin{enumerate}
    \item \textbf{Universality}: How broadly does dimension matching hold beyond GMC? What are the minimal conditions?

    \item \textbf{Critical behavior}: What happens exactly at the transition? Can the breakdown be characterized precisely?

    \item \textbf{Control}: Can dimension matching be restored by intervention? What are the minimal control inputs?

    \item \textbf{Higher-order matching}: Are there additional complexity measures that should also agree? What is the full ``matching class''?
\end{enumerate}

\subsection{Conclusion}

Dimension matching—the coincidence of geometric and spectral complexity measures—is not an accident but a signature of coherent multiscale structure. Its presence indicates that different observations of the system yield consistent complexity estimates; its absence signals impending collapse. The martingale mechanism underlying matching can be interpreted as ``fairness'' across scales, naturally framed as a cooperative equilibrium in a multi-level coordination game.

For practitioners, this provides a diagnostic: compute correlation and harmonic dimensions from time series data and compare. Agreement suggests robustness; divergence warns of transition. For theorists, the GMC framework offers a rigorous template for what ``coherent chaos'' means—and the mathematics to characterize its limits.

%==============================================================================
% REFERENCES
%==============================================================================
\bibliography{references}

\end{document}
